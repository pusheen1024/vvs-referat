\documentclass[10pt]{article}

\usepackage{preamble}

\begin{document}

\section*{Олимпиадное программирование. Выступление}
\begin{enumerate}
\item Представиться, назвать тему.
\item Олимпиадные задачи не похожи на задачи промышленной разработки. Почему же олимпиадное программирование нужно и важно? Важность олимпиадного программирования для меня лично, почему выбрала эту тему, цель работы.
\item Как выглядит условие олимпиадной задачи, на что обратить внимание (легенда, формат входных и выходных данных, примеры, ограничения).
\item Вердикты: TLE, MLE, AC. IOI"=style и ICPC"=style (кратко).
\item <<Codeforces "--- градообразующее предприятие Саратова>>.
\item Функционал Codeforces. Раунды, система рейтинга, социальная сеть.
\item Школьные олимпиады: ВСОШ, IOI (успехи российских школьников), ВКОШП.
\item ICPC: первый финал проводился 2 февраля 1977 в Атланте, масштаб соревнования с каждым годом лишь растёт (111 стран, 60000 студентов).
\item Этапы ICPC, другие студенческие олимпиады.
\item Олимпиадное программирование в СГУ: 1998 "--- СГУ стал организатором ЧФ Поволжья и Юга России (27 субъектов РФ). 2002 "--- первый финал, 6 место в мире.
\item 2006 "--- СГУ чемпионы мира, 2008 "--- чемпионы России.
\item ЦОПП: основан 23 июля 2003 года, 24 июля ушла из жизни Наталья Львовна Андреева, чьё имя было присвоено Центру. Руководители ЦОППа: 2009--2018 "--- Михаил Мирзаянов, 2003--2009 и 2018--н.в. "--- Антонина Гавриловна Фёдорова.
\item ЦОПП сегодня: тренировки, лекции, новый набор.
\item Тренеры и руководители ЦОППа сегодня.
\item Локальные олимпиады: внутривуз, межвуз "--- подготовлены сотрудниками университета.
\item СГУ на ICPC сегодня: квалы+ЧФ (многие принимали участие), полуфинал, финал.
\item Сборы: Дубки, Речная Долина. Атмосфера, единомышленники, путешествия, еда, мерч, новые перспективы и возможности.
\item Спасибо за внимание!
\end{enumerate}

\newpage
Многие задаются вопросом: зачем вообще нужно олимпиадное программирование? И действительно, олимпиадные задачи могут показаться непохожими на задачи промышленной разработки и даже несколько оторванными от реальности. Несмотря на это, олимпиадное программирование имеет огромное количество преимуществ. Оно учит нас писать код в условиях ограниченного времени, преодолевать волнение, не сдаваться, соревноваться и побеждать. Умение быстро думать и усваивать материал, работать в команде может быть полезно и в дальнейшей карьере. Я сама активно занимаюсь олимпиадным программированием и именно поэтому решила подробнее рассмотреть эту тему.

При решении олимпиадной задачи важно обратить внимание на формат входных и выходных данных: решение, выводящее даже правильный ответ, но в некорректном формате, зачтено не будет. Помимо этого также важны ограничения на входные данные: нижние ограничения важны для обработки крайних случаев, а верхние нередко указывают на требуемую асимптотику. В задаче всегда есть ограничения по времени и памяти, если решение в них не укладывается, оно <<падает>> с вердиктом TLE/MLE. Существует две системы оценивания решений: ICPC"=style и IOI"=style. В первом случае решение тестируется до первого упавшего теста и засчитывается, только если пройдены все тесты, во втором же случае за подгруппы начисляются частичные баллы.

Существует большое количество платформ для решения олимпиадных задач, наиболее популярной из которых является Codeforces. Один мой знакомый однажды сказал, что Codeforces "--- градообразующее предприятие Саратова, и действительно, этот сайт был создан в 2010 году тогдашним тренером команд СГУ Михаилом Мирзаяновым. На Codeforces регулярно проводятся раунды "--- короткие соревнования по программированию, также существует система рейтинга: в зависимости от выступления на раунде он повышается или понижается. Codeforces предоставляет и другой функционал: можно дорешивать задачи в архиве, создавать тренировки и группы, публиковать блоги.

Для многих путь в олимпиадном программировании начинается ещё в школе. Самой важной олимпиадой для школьников является ВСОШ, которая также служит отбором на IOI, где российские школьники из года в год показывают высокие результаты. Некоторые школьные олимпиады дают возможность поступить в университет БВИ (я, к примеру, так и сделала). Проводятся и командные олимпиады, например, ВКОШП, хоть они и не так распространены среди школьников, как среди студентов.

Самой важной олимпиадой для студентов, безусловно, является ICPC "--- международный командный чемпионат по спортивному программированию. Впервые финал ICPC был проведён в 1977 году в Атланте, и с тех пор масштаб соревнования лишь растёт. У нас он состоит из 4 этапов: квалификация, четверть"=финал, полуфинал и международный финал. Помимо этого, существуют и другие студенческие олимпиады, которые организуют различные университеты.

Отдельно нужно сказать об истории олимпиадного программирования в СГУ, ведь наш университет "--- один из первых в России, где зародилось олимпиадное движение. В 1998 году СГУ стал организатором четверть"=финала Поволжья и Юга России и является им до сих пор. Уже в 2002 году студенты СГУ отправились на финал ICPC на Гавайи, где заняли 6ое место в мире. В 2006 году произошло значимое для нашей истории событие: команда университета стала чемпионами мира! В 2008 же году другая команда СГУ смогла стать чемпионами России.

Огромную роль в подготовке студентов к соревнованиям играет ЦОПП. Приказ о создании Центра был подписан 23 июля 2003 года, а 24 июля ушла из жизни Наталья Львовна Андреева, идейный вдохновитель олимпиадного движения в Саратове, и созданному Центру было по праву присвоено её имя. Некоторое время руководителем ЦОППа был Михаил Мирзаянов, основатель Codeforces, а сейчас им является Антонина Гавриловна Фёдорова.

В настоящее время в ЦОППе каждую неделю проводятся лекции и тренировки. Ежегодно проводится организационное собрание <<нового набора>> и начинаются лекции для базовой группы, которые, как я думаю, посещают многие из присутствующих. На этой фотографии вы можете увидеть тренеров и руководителей ЦОППа. У нас регулярно проводятся олимпиады по программированию, например, внутривузовская и межвузовская, задачи для которых готовят сотрудники Центра. Конечно, университет продолжает участвовать и в ICPC. В этом году 4 команды, в том числе и наша, ездили в Санкт"=Петербург на полуфинал ICPC, а ведущая команда будет представлять университет на финале в сентябре в Баку.

В заключение я бы хотела сказать, что ЦОПП "--- это место с душевной атмосферой, где всегда можно найти единомышленников. Каждый год у нас проводятся сборы, где мы не только пишем контесты, но и весело проводим время. У нас есть возможность путешествовать и бывать в новых местах благодаря олимпиадам. Компании также заинтересованы в студентах"=олимпиадниках, и поэтому не только спонсируют мероприятия и раздают мерч, но и приглашают студентов на стажировки уже даже на первом курсе. Олимпиадное программирование открывает перед нами множество новых перспектив и возможностей. Спасибо за внимание!


\end{document}

